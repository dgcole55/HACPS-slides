\documentclass{article}

\usepackage{longtable}
\usepackage{booktabs}
\usepackage{amsmath}

\begin{document}

\section{Missionaries and Cannibals}\label{missionaries-and-cannibals}

Three missionaries and three cannibals must cross a river using a boat
which can carry at most two people, under the constraint that, for both
banks, if there are missionaries present on the bank, they cannot be
outnumbered by cannibals (if they were, the cannibals would eat the
missionaries). The boat cannot cross the river by itself with no people
on board.

\begin{center}\rule{0.5\linewidth}{0.5pt}\end{center}

\section{Some math}\label{some-math}

\begin{itemize}

\item
  \(\mathcal{M}\), the set of missionaries
\item
  \(\mathcal{C}\), the set of cannibals
\item
  \(\mathcal{B} = \{ E, W \}\), the banks of the river
\end{itemize}

It is helpful to realize that everyone is the union of \(\mathcal{M}\)
and \(\mathcal{C}\).

\[\mathcal{E} = \mathcal{M} \cup \mathcal{C}\]

\begin{center}\rule{0.5\linewidth}{0.5pt}\end{center}

\section{Describing the boat and who is on each
bank}\label{describing-the-boat-and-who-is-on-each-bank}

\begin{itemize}

\item
  \(b\) is the bank where the boat is
\end{itemize}

\[b \in \mathcal{B} = \{ E, W \}\]

\begin{itemize}

\item
  \(o(b)\), the other bank
\end{itemize}

\begin{longtable}[]{@{}ll@{}}
\toprule\noalign{}
\(b\) & \(o(b)\) \\
\midrule\noalign{}
\endhead
\bottomrule\noalign{}
\endlastfoot
\(E\) & \(W\) \\
\(W\) & \(E\) \\
\end{longtable}

\begin{center}\rule{0.5\linewidth}{0.5pt}\end{center}

\section{Who is on each bank}\label{who-is-on-each-bank}

\begin{itemize}

\item
  \(w\) is who is on a bank. There are wwo way to think of this:
\end{itemize}

\begin{enumerate}
\def\labelenumi{\arabic{enumi}.}

\item
  \(w\) is a function from \(\mathcal{B}\) to some subset of everyone
  \(\mathcal{M} \cup \mathcal{C}\)
\end{enumerate}

\[ w : \mathcal{B} \rightarrow  \subset \mathcal{M} \cup \mathcal{C}\]

\begin{enumerate}
\def\labelenumi{\arabic{enumi}.}
\setcounter{enumi}{1}

\item
  \(w\) is an array indexed by \(E\) and \(W\) telling you who is on
  each bank.
\end{enumerate}

\begin{longtable}[]{@{}ll@{}}
\toprule\noalign{}
\(x\) & \(w(x)\) \\
\midrule\noalign{}
\endhead
\bottomrule\noalign{}
\endlastfoot
\(E\) & \(S\) \\
\(W\) & \((\mathcal{M} \cup \mathcal{C})\setminus S\) \\
\end{longtable}

\begin{center}\rule{0.5\linewidth}{0.5pt}\end{center}

\section{Is it safe}\label{is-it-safe}

Let \(S\) be a subset of people on a bank of the river (presumably
gotten from \(w\)).

\begin{itemize}

\item
  The people on the bank can just be cannibals,
\end{itemize}

\[S \subseteq \mathcal{C}\]

\begin{itemize}

\item
  or, the number of cannibals in \(S\) must be less than or equal to the
  number of missionaries.
\end{itemize}

\[|S \cap C| \leq |S \cap M|\]

We can write this as a function \$ \text{IsSafe} : S
\rightarrow \text{boolean}\$.

\[\text{IsSafe}(S) = \left( S \subseteq \mathcal{C} \right) \wedge ( |S \cap C| \leq |S \cap M| )\]

\begin{center}\rule{0.5\linewidth}{0.5pt}\end{center}

\section{An initial condition}\label{an-initial-condition}

Starting out - \(b = E\) - Who is on the banks

\begin{longtable}[]{@{}ll@{}}
\toprule\noalign{}
bank & who \\
\midrule\noalign{}
\endhead
\bottomrule\noalign{}
\endlastfoot
\(E\) & \(\mathcal{M} \cup \mathcal{C}\) \\
\(W\) & \(\emptyset\) \\
\end{longtable}

\begin{center}\rule{0.5\linewidth}{0.5pt}\end{center}

\section{\texorpdfstring{We want to move the people in \(S\) from bank
\(b\) to the other bank
\(o(b)\)}{We want to move the people in S from bank b to the other bank o(b)}}\label{we-want-to-move-the-people-in-s-from-bank-b-to-the-other-bank-ob}

Let's define two things

\begin{enumerate}
\def\labelenumi{\arabic{enumi}.}

\item
  After the move the new set of people \(N\) is who is on the bank minus
  who went on the boat
\end{enumerate}

\[N = w(b) \setminus S \tag{new on this bank}\]

\begin{enumerate}
\def\labelenumi{\arabic{enumi}.}
\setcounter{enumi}{1}

\item
  After the move the new set of people on the other bank is who is on
  the other bank plus who went on the boat
\end{enumerate}

\[O = w(o(b)) \cup S \tag{new on other bank}\]

\begin{center}\rule{0.5\linewidth}{0.5pt}\end{center}

\section{What is a valid move?}\label{what-is-a-valid-move}

Several things must be true:

\begin{enumerate}
\def\labelenumi{\arabic{enumi}.}

\item
  Only two people can be in the boat
\end{enumerate}

\[|S| \in \{ 1, 2 \} \tag{number in boat}\]

\begin{enumerate}
\def\labelenumi{\arabic{enumi}.}
\setcounter{enumi}{1}

\item
  The new set of people on the bank must be safe
\end{enumerate}

\[\text{IsSafe}(N) = \text{true}\]

\begin{enumerate}
\def\labelenumi{\arabic{enumi}.}
\setcounter{enumi}{2}

\item
  The new set of people on the other bank must be safe
\end{enumerate}

\[\text{IsSafe}(O) = \text{true}\]

\begin{enumerate}
\def\labelenumi{\arabic{enumi}.}
\setcounter{enumi}{3}

\item
  The boat is now on the other bank
\end{enumerate}

\[b' = o(b)\]

\begin{enumerate}
\def\labelenumi{\arabic{enumi}.}
\setcounter{enumi}{4}

\item
  Who is on each bank is
\end{enumerate}

\[w'(b) = N \quad \text{and} \quad w'(o(b)) = O\]

\begin{center}\rule{0.5\linewidth}{0.5pt}\end{center}

\section{In math terms}\label{in-math-terms}

\[
\begin{array}{rl}
\text{ValidMove(S,b)} & = \text{number in boat} \\
    & \wedge\quad \text{IsSafe}(N) \\
    & \wedge \quad \text{IsSafe}(O) \\
    & \wedge \quad b' = o(b) \\
    & \wedge \quad w'(b) = N \\
    & \wedge \quad w'(o(b)) = O 
\end{array}
\]

\begin{center}\rule{0.5\linewidth}{0.5pt}\end{center}

\section{Finally, we know that at any step, there must be some group of
people on the bank where the canoe is that result in a valid
move}\label{finally-we-know-that-at-any-step-there-must-be-some-group-of-people-on-the-bank-where-the-canoe-is-that-result-in-a-valid-move}

\[\text{ValidNext} = \exists S \subset w(b) \; : \;  \text{ValidMove}(S,b) = \text{true}\]

\end{document}
